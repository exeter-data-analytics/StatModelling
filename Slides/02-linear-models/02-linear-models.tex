%=============================================================================%
% Author: 	John Joseph Valletta
% Date: 	07/11/2018
% Title: 	Fews slides for linear modelling workshop
%=============================================================================%

%=============================================================================%
% Preamble
%=============================================================================%
% Libraries
\documentclass[pdf]{beamer}
\usepackage[export]{adjustbox}
\usepackage{framed}
\usepackage{color}
\definecolor{dkgreen}{rgb}{0,0.6,0}
\definecolor{gray}{rgb}{0.5,0.5,0.5}
\definecolor{mauve}{rgb}{0.58,0,0.82}
\definecolor{deepblue}{rgb}{0,0,0.5}
\definecolor{deepred}{rgb}{0.6,0,0}
\definecolor{deepgreen}{rgb}{0,0.5,0}
\definecolor{lightgray}{rgb}{0.92,0.92,0.92}
\usepackage{listings} % to insert code
\usepackage{textpos} % textblock
\usepackage{verbatim}
\usepackage{hyperref}
\hypersetup{colorlinks=true, urlcolor=blue, linkcolor=black} 

% Listing set up
\lstdefinestyle{R}{
language=R,                     % the language of the code
basicstyle=\scriptsize\ttfamily,% the size of the fonts that are used for the code
numbers=left,                   % where to put the line-numbers
numberstyle=\tiny\color{gray},  % the style that is used for the line-numbers
stepnumber=1,                   % the step between two line-numbers. If it's 1, each line
                          		% will be numbered
numbersep=5pt,                  % how far the line-numbers are from the code
backgroundcolor=\color{lightgray},  % choose the background color. You must add \usepackage{color}
showspaces=false,               % show spaces adding particular underscores
showstringspaces=false,         % underline spaces within strings
showtabs=false,                 % show tabs within strings adding particular underscores
frame=lines,%single,                   % adds a frame around the code
rulecolor=\color{black},        % if not set, the frame-color may be changed on line-breaks within not-black text (e.g. commens (green here))
tabsize=2,                      % sets default tabsize to 2 spaces
captionpos=b,                   % sets the caption-position to bottom
breaklines=true,                % sets automatic line breaking
breakatwhitespace=false,        % sets if automatic breaks should only happen at whitespace
title=\lstname,                 % show the filename of files included with \lstinputlisting;
                          		% also try caption instead of title
keywordstyle=\color{blue},      % keyword style
commentstyle=\color{dkgreen},   % comment style
stringstyle=\color{mauve},      % string literal style
escapeinside={\%*}{*)},         % if you want to add a comment within your code
morekeywords={}            		% if you want to add more keywords to the set
}

% Presentation configuration
\mode<presentation>{\usetheme{Madrid}}
\definecolor{tealblue}{rgb}{0, 0.5, 0.5}
\usecolortheme[named=tealblue]{structure}
\useinnertheme{circles} % circles, rectanges, rounded, inmargin
\usefonttheme[onlymath]{serif} % makes math fonts like the usual LaTeX ones
\setbeamercovered{transparent=4} % transparent
\setbeamertemplate{caption}{\raggedright\insertcaption\par} % Remove the word "Figure" from caption %\setbeamertemplate{caption}[default]
\setbeamertemplate{navigation symbols}{} % don't put navigation tools at the bottom (alternatively \beamertemplatenavigationsymbolsempty)
\graphicspath{ {./img/} }

% Titlepage
\title[Linear models in R]{Linear models in R}
%\subtitle{Built-in data types}
\author{John Joseph Valletta}
\date[November 2018]{November 2018}
\institute[]{University of Exeter, Penryn Campus, UK}
\titlegraphic{
\hfill
\includegraphics[width=\textwidth, keepaspectratio]{logo.jpg}}

%=============================================================================%
%=============================================================================%
% Start of Document
%=============================================================================%
%=============================================================================%
\begin{document}

%=============================================================================%
%=============================================================================%
\begin{frame}
\titlepage
\end{frame}

%=============================================================================%
%=============================================================================%
\begin{frame}{What is a model and why do we need one?}

\visible<1->{
\begin{block}{}
A \textbf{model} is a human construct/abstraction that tries to approximate the \textbf{data generating process} in some useful manner
\end{block}}

\vfill

\visible<2->{
\begin{block}{Models are built for}
	\begin{itemize}
		\item<3-> enhancing our understanding of a complex phenomenon
		\item<4-> executing \href{https://www.youtube.com/watch?v=owI7DOeO_yg}{``what if"} scenarios
		\item<5-> predicting/forecasting an outcome
		\item<6-> controlling a system
	\end{itemize}
\end{block}}

\end{frame}

%=============================================================================%
%=============================================================================%
\begin{frame}{Illustrative example}

\begin{figure}
\includegraphics<1>[width=.65\textwidth]{example1.pdf}
\includegraphics<2>[width=.65\textwidth]{example2.pdf}
\includegraphics<3>[width=.65\textwidth]{example3.pdf}
\includegraphics<4>[width=.65\textwidth]{example4.pdf}
\end{figure}

\end{frame}

%=============================================================================%
%=============================================================================%
\begin{frame}{Formal definition}

$$
\begin{aligned}
y_i & = \beta_0 + \beta_1x_i + \epsilon_i \\
\epsilon_i & \sim \mathcal{N}(0, \sigma^2)
\end{aligned}
$$

\vfill

\visible<2->{
\begin{block}{Observed data}
\begin{itemize}
    \item $y$ {\scriptsize (outcome/response)}: minutes spent talking about GoT
    \item $x$ {\scriptsize(explanatory)}: hours spent watching Game of Thrones (GoT)
\end{itemize}
\end{block}}

\vfill

\visible<3->{
\begin{block}{Parameters to infer}
\begin{itemize}
    \item $\beta_0$: intercept
    \item $\beta_1$: gradient wrt minutes talking about GoT
\end{itemize}
\end{block}}

\end{frame}

%=============================================================================%
%=============================================================================%
\begin{frame}[fragile]
\frametitle{Linear models in R}

\begin{itemize}\addtolength{\itemsep}{0.5\baselineskip}
    \item Use the \texttt{lm()} function
    \item Requires a \textbf{formula} object\\ \texttt{outcome $\sim$ explanatory variable}
\end{itemize}

\vfill

\begin{lstlisting}[style=R]
# talk: minutes spent talking about GoT (outcome/response variable)
# watch: hours spent watching GoT (explanatory variable) 

fit <- lm(talk ~ watch)

# If data is in a data frame called "df"
fit <- lm(talk ~ watch, df) 
\end{lstlisting}

\end{frame}

%=============================================================================%
%=============================================================================%
\begin{frame}[fragile]
\frametitle{Summary of fitted model}

\begin{lstlisting}[style=R]
summary(fit)
\end{lstlisting}

\scriptsize
\begin{verbatim}
## 
## Call:
## lm(formula = height ~ weight, data = df)
## 
## Residuals:
##     Min      1Q  Median      3Q     Max 
## -31.089  -6.926  -0.689   6.057  24.967 
## 
## Coefficients:
##             Estimate Std. Error t value Pr(>|t|)    
## (Intercept)  2.35229    7.11668   0.331    0.742    
## weight       2.17446    0.08782  24.762   <2e-16 ***
## ---
## Signif. codes:  0 '***' 0.001 '**' 0.01 '*' 0.05 '.' 0.1 ' ' 1
## 
## Residual standard error: 10.31 on 98 degrees of freedom
## Multiple R-squared:  0.8622, Adjusted R-squared:  0.8608 
## F-statistic: 613.1 on 1 and 98 DF,  p-value: < 2.2e-16
\end{verbatim}
\normalsize

\end{frame}

%=============================================================================%
%=============================================================================%
\begin{frame}[fragile]
\frametitle{Model checking}

In order to make \textbf{robust} inference, we must check the model fit

\vspace{0.5cm}

\begin{lstlisting}[style=R]
plot(fit)
\end{lstlisting}

\vspace{-0.5cm}

\begin{figure}
\includegraphics<1>[width=.5\textwidth]{check.pdf}
\end{figure}

\end{frame}

%=============================================================================%
%=============================================================================%
\begin{frame}{Confidence vs prediction intervals}

\begin{figure}
\includegraphics<1>[width=.6\textwidth, height=0.6\textwidth]{confInt.pdf}
\includegraphics<2>[width=.6\textwidth, height=0.6\textwidth]{predInt.pdf}
\end{figure}

\end{frame}

%=============================================================================%
%=============================================================================%
\begin{frame}{Multiple linear regression}

$$
\begin{aligned}
y_i & = \beta_0 + \beta_1x_{1i} + \ldots + \beta_px_{pi} + \epsilon_i \\
\epsilon_i & \sim \mathcal{N}(0, \sigma^2)
\end{aligned}
$$

\begin{figure}
\includegraphics<1>[width=.5\textwidth]{multiple1.pdf}
\includegraphics<2>[width=.5\textwidth]{multiple2.pdf}
\end{figure}

\end{frame}

%=============================================================================%
%=============================================================================%
\begin{frame}{Categorical variables}

\begin{figure}
\includegraphics<2->[width=.6\textwidth]{categorical.pdf}
\end{figure}

\end{frame}

%=============================================================================%
%=============================================================================%
\begin{frame}{Categorical variables}

We need \textbf{dummy} variables

$$
S_i = \left\{\begin{array}{ll}
        1 & \mbox{if $i$ is male},\\
        0 & \mbox{otherwise}
        \end{array}
        \right.
$$

Here, female is known as the \textbf{baseline/reference level}

The regression is:

$$
y_i = \beta_0 + \beta_1 S_i + \beta_2 x_i + \epsilon_i
$$

Or in English:

$$
\begin{aligned}
\mathrm{height}_i & = \beta_0 + \beta_1\mathrm{sex}_i + \beta_2\mathrm{weight}_i + \epsilon_i \\
\end{aligned}
$$

\end{frame}

%=============================================================================%
%=============================================================================%
\begin{frame}{Categorical variables}

The mean regression lines for male and female are:

\begin{itemize}\addtolength{\itemsep}{2\baselineskip}

\item Female (\texttt{sex=0})

$$
\begin{aligned}
    \mathrm{height}_i & = \beta_0 + (\beta_1 \times 0) + \beta_2\mathrm{weight}_i\\
    \mathrm{height}_i & = \beta_0 + \beta_2\mathrm{weight}_i
\end{aligned}
$$

\item Male (\texttt{sex=1})

$$
\begin{aligned}
    \mathrm{height}_i & = \beta_0 + (\beta_1 \times 1) + \beta_2\mathrm{weight}_i\\
    \mathrm{height}_i & = (\beta_0 + \beta_1) + \beta_2\mathrm{weight}_i
\end{aligned}
$$

\end{itemize}

\end{frame}

%=============================================================================%
%=============================================================================%
% End of Document
%=============================================================================%
%=============================================================================%
\end{document}
